\documentclass[12pt]{article} % Класс документа, шрифт 12pt

% Пакеты
\usepackage[utf8]{inputenc} % Кодировка UTF-8
\usepackage[russian]{babel} % Русский язык

% Начало документа
\begin{document}

\section{Краткое описание микрокалькулятора}


Микрокалькулятор "Электроника Б3-36" предназначен для научных расчетов.

Микрокалькулятор автоматически выполняет четыре арифметических действия, вычисления натуральных и десятичных логарифмов и антилогарифмов, прямых и обратных тригонометрических функций, обратных величин, факториала, вычисления с двухуровневыми скобками, возведение в степень, извелечение корней и операции с памятью.

Ввод данных и команд в микрокалькуляторе осуществляется вручную с помощью клавиатуры.

Наличие клавиши совмещенной функции F позволяет использовать клавиши для выполнения двух операций. Обозначение второй функции расположено над клавишами. Контроль ввода данных и результатов вычислений производится визуально с помощью 12-знакоместного катодолюминесцентного индикатора. Вычислительное устройство микрокалькулятора выполнена на одной микросхеме.

Микрокалькулятор может находится в одном из режимов: основном, совмещенной функции, переполнения. Микрокалькулятор воспроизводит исходные данные и результаты вычислений в экспоненциальной форме или с естественной запятой.

Пример индикации числа, выраженного в экспоненциальной форме, представленный ниже, читается следующим образом: 

\begin{center}
$-3,14159226 * 10^{-79}.$
\end{center}


{\centering Назначение клавиш \par}
\vspace{0.25cm}

\textbf{В основном режиме}
\vspace{0.25cm}

\textbf{Клавиши ввода}
\vspace{0.5cm}

\begin{tabular}{p{2cm} p{9cm}}
0 ... 9 & - Цифровые клавиши; \\
.       & - Клавиша десятичной запятой; \\
\end{tabular}

\newpage

\textbf{Клавиши операционные}
\vspace{0.5cm}

\begin{tabular}{ p{2cm} p{9cm} }
+ - х ÷ & - Клавиши арифметических операций;\\
= & - Клавиша выполнения операций;\\ 
$[(  )]$ & - Клавиша вычислений с двухуровневыми скобками;\\
/-/ & - Клавиша смены знака числа и его порядка;\\
<-> & - Клавиша обмена между содержимым регистра констант и регистра индикации;\\
$\pi$ & - Клавиша ввода числа $\pi$ (на индикаторе отображается число 3,1415926);\\
ВП & - Клавиша ввода порядка числа.\\
C & - Клавиша сброса\\
\end{tabular}
\vspace{0.5cm}

В том случае, если микрокалькулятор находится в режиме совмещенной функции, для возрата в основной режим и очистки регистров микрокалькулятора необходимо трехкратное нажатие клавиши С . Первое нажатие снимает режим совмещенной функции, второе - очищает регистр индикации, третье - очищает регистр констант. Если микрокалькулятор находится в режиме переполнения, для возрата в основной режим и очистки регистров микрокалькулятора необходимо однократное нажатие С .
При нажатии клавиши С . не происходит очистки регистра памяти.
Примечание. При нажатии любой из клавиш + , - , х, ÷ выполняется ранее установленная операция (если такая имеется), замет микрокалькулятор подготавливается к выполнению веденной операции.

\newpage
\textbf{В режиме совмещенной функции}

Перевод микрокалькулятора в режим совмещенной функции осуществляется нажатием клавиши F . Клавиша ARC обеспечивает вычисление обратных тригонометрических функций при последующем нажатии клавиш  sin , cos , tg .

\begin{tabular}{ p{3.5cm} p{9cm} }
$y^x$ & - Клавиша возведения числа в степень; \\
$\sqrt{x}$ & - Клавиша извелечение квадратного корня; \\
1/x & - Клавиша вычисления обратной величины числа; \\
n! & - Клавиша вычисления факториала; \\
sin , cos, tg & - Клавиша вычислений тригонометрических функций; \\
lg, ln , $10^x$, $e^x$ & - Клавиши вычисления десятичных и натуральных логарифмов и антилогарифмов; \\
П+ , П-, Пх, П÷ & - Клавиши операций с памятью; \\
Г->Р  Р->Г & - Клавиши перевода величин, выраженных в градусах, в радианы и обратно; \\
X<->П & - Клавиша обмена между содержимым регистра индикации и регистра памяти; \\
ИП & - Клавиша вызова содержимого регистра индикации; \\
ЗП & - Клавиша записи в регистр памяти содержимого регистра индикации; \\
СП & - Клавиша очистка регистра памяти; \\
CF & - Клавиша снятия режима совмещенной функции; \\
\end{tabular}


Примечание. При выполнении операций "П+", "П-", "Пх", "П÷" число на индикаторе не изменяется, результат вычисления хранится в регистре памяти.
Операция очистки регистра памяти не влияет на содержимое регистра индикации или заданную операцию.

\vspace{0.5cm}
\textbf{В режиме переполнения}

Режим переполнения возникает, если:

- Любой результат вычислений или промежуточный ответ превышает $9.99999999 * 10^99$ или менее $10^-99$ независимо от арифметического знака;

- Результат операции с памятью превышает $9.99999999 * 10^99$ или менее $10^-99$ независимо от арифметического знака (предыдущее содержимое регистра памяти будет заменено числом, вызвавшим переполнение);

- Производится операция деления на "0";

- Производится вычисления функций от аргументов, значения которых выходят за пределы их области определния.

При этом на индикаторе высвечиваются 10 значащих цифр или 10 нулей и точки во всех разрядах.

Для снятия режима переполнения необходимо однократное нажатие клавиши С .

Примечание. Индикация переполнения может исчезнуть после нажатия на другие клавиши, но микрокалькулятор не будет возращен в основной режим и последующие вычисления будут ошибочными.
%\section{Технические данные}
%Выполненные операции        Четыре арифметических действия, двухуровневые скобочные вычисления, вычисления натуральных и десятичных логарифмов и антилогарифмов, прямых и обратных тригонометрических функций, извелечение квадратного корня и возведение в степень любых действительных чисел, вычисление обратной величины числа и перевод величин, выраженных в градусах, в величины, выраженные в радианах, и обратно, запоминания данных и операции с памятью.

\section{Примеры вычислений}

\begin{tabular}{ p{5cm} p{4cm} p{4cm} }
    1. Сложение, вычитание, умножение, деление \\
    $\frac{(3+7)*5-3}{8} = 5,875$ & 3 + 7 x 5 - 3 ÷ 8 = & 5 . 8 7 5 \\ 
    \vspace{0.1cm}

    2. Цепочные операции \\
    $8+5+3+3+3 = 22$ & 8 + 5 + 3 = = = & 2 2 \\
    \vspace{0.1cm}

    3. Операции с постоянным множителем: \\
    $8 * 2 = 16$ & 8 x 2 = & 1 6. \\ 
    $5 * 2 = 10$ & 5 = & 1 0. \\
    $17 * 2 = 34$ & 17 = & 3 4. \\
    \vspace{0.1cm}

    4. Операции с двухуровневыми скобками: \\
    $\frac{(15 * 4) + (7 + 8)}{(3 + 2) * (4 - 6)} = -7,5$ & $[($ $[($ 15 x 4 $)]$ & 6 0. \\ 
                                                          & + $[($ 7 + 8 $)]$ $)]$ & 7 5. \\ 
                                                          & ÷ $[($ $[($ 3 + 2 $)]$ & 5. \\ 
                                                          & x $[($ 4 - 6 $)]$ $)]$ = & 7. 5\\
    \vspace{0.1cm}

    5. Смена знака числа: \\
    $\frac{8 * (-7)}{7} = -8$ & 8 x 7 /-/ ÷ 7 = & - 8 \\
    \vspace{0.1cm}

    6. Ввод числа в экспоненциальной форме: \\
    $8 * 10^{-6} * 7 = 0,000056$ & 8 ВП 6 /-/ x 7 = & 5. 6 0 0 0 0 0 0 - 5 \\
    \vspace{0.1cm}

    7. Операция с константой $\pi$ : \\
    $3 * \pi = 9,4247778$ & 3 x $\pi$ = & 9. 4 2 4 7 7 7 8 \\ 
    \vspace{0.1cm}

    8. Обмен содержимого регистров индикации и констант : \\
    $\frac{16}{2+1} = 5,333333$ & 2 + 1 ÷ 16 <-> = & 5. 3 3 3 3 3 3 3 \\
\end{tabular}

\begin{tabular}{ p{5cm} p{4cm} p{4cm} }
    9. Операция с использованием регистра памяти: \\
    $\frac{9}{12} + \frac{5}{34} - \frac{11}{18} = 0,285947$ & 9 ÷ 12 = F ЗП  5 ÷ 34 = F П+  11 ÷ 18 = F П-  F ИП & 2. 8 5 9 4 7 7 1 \\
    \vspace{0.1cm}

    10. Перевод градусной меры в радианную и обратно: \\
    $\pi$ радиан $ = 180^{\circ} = \pi $ радиан & $\pi$ F  Р->Г & 1 7 9. 9 9 9 9 9 \\
    F  Г->Р  & 3. 1 4 1 5 2 5 \\
    \vspace{0.1cm}
    
    11. Вычисление тригонометрических функций: \\ 
    $\sin 30^{\circ} = 0.5$ & Град. 30  F  sin & 5 \\
    $\cos 45^{\circ} = 0.70717$ & Град. 45  F  cos  & 7. 0 7 1 7 \\
    $\tg 1.1 Радиан = 45^{\circ} $ & Рад. 1. 1  F  tg  & 1. 9 6 4 7 6 \\ 
    \vspace{0.1cm}

    12. Вычисление обратных тригонометрических функций: \\
    $\arcsin 0.389 = 22.8923$ & . 389 ARC  sin & 2 2. 8 9 2 3 \\
    $\arccos 0.8660254 = 30$ & . 8660254 ARC  cos & 3 0. \\
    $\arctg 1 = 45$ & 1 ARC  tg & 4 5. \\
    \vspace{0.1cm}

    13. Извлечение корней и возведение в степень: \\  
    $36 * \sqrt{4} = 72$ & 36 x 4  F  $sqrt{x}$ = & 7 2. \\
    $4 * 5^{3,5} =$ & 4 x [( 5  F  $y^x$ 3 . 5  )]  = & 1 1 1 8. 0 3 2 \\
    $32^{frac{1}{5} = 2}$ & 32  F  $y^x$  5  F 1/x = & 2 \\
    \vspace{0.1cm}

    14. Вычисление обратной величины: \\
    $\frac{1}{\frac{1}{3} + \frac{1}{10} + \frac{1}{30}} = 2,148571$ & 3  F 1/x + 10  F  1/x & \\
                                                                     & +  30  F  1/x  =  F  1/x & 2. 1 4 2 8 5 7 1 \\
    \vspace{0.1cm}

\end{tabular}

\begin{tabular}{ p{5cm} p{4cm} p{4cm} }
    15. Вычисление натурального и десятичного логарифмов: \\
    $lg 120 = 2,07918$ & 120  F  $lg$ & 2. 0 7 9 1 8 \\
    $ln (32)^3 = 10,39722$ & 3 x 32  F  $ln$  =  1 0. 3 9 7 2 2 \\
    \vspace{0.1cm}

    16. Вычисление антилогарифмов: \\
    $10^(2,32) = 208,93$ & 2 . 32  F  $10^x$ &  2 0 8. 9 3 \\
    $e^(-3) = 0,0497871$ & 3 /-/   F  $e^x$  &  4. 9 7 8 7 1 \\
    \vspace{0.1cm}

    17. Операции с вычислением факториала: \\
    $c^{6}_{10} = \frac{10!}{6!(10-6)!} = 210$ & 10  F  $n!$ ÷ [( [( \\
                                               & 6  F  )]  x  [( & \\ 
                                               & 10 - 6 )]  F  $n!$ & \\
                                               & )] )]  = & 2 1 0. \\
    \vspace{0.1cm}

\end{tabular}

\end{document}

